\begingroup
\newcommand\blockA[1]{\parbox[c][0.23\textheight][c]{0.3\textwidth}{\centering #1}}
\newcommand\blockB[1]{\parbox[c][0.23\textheight][c]{0.65\textwidth}{\raggedright #1}}

% Magdalene Brahe
\blockA{\adjincludegraphics[height=0.27\textheight]{Bilder/Karaktarer/Nabla.png}}
\blockB{\textsc{Nabla}\\
%Magdalene är dotter till Tycho och en baddare på mätteknik. Hon är glad och livlig men tar jobbet och studierna som astronom väldigt seriöst. Hon drömmer om kärlek, men självklart på sina egna villkor.}
Nabla är tvillingsyster till Laplace. Otrolig intelligens, övermänsklig charm, magnifik fysik. Alla dessa och fler är egenskaper som hon inbillar sig att hon besitter. Cybernetisk augmentation är hennes passion och hon missar aldrig en chans att förbättra sin kropp med ny teknik, vilket också är anledningen till att hon hjälper Mr. Andersson med hans forskning. Behöver kanske inte nämnas att hon också är en cyborg.}
% Giuseppe Arcimboldo
\vfil
\blockB{\vspace{2ex}\textsc{Mr. Andersson}\\
%Giuseppe är hovkonstnär vid det kejserliga hovet i Prag. Han är bekväm med sitt jobb, där han målar porträtt varje dag, och är bästa vän med kejsare Rudolf II. Han anser sig själv vara den bästa och vackraste konstnären i hela den vida världen, men hans konststil skulle kunna beskrivas som väldigt enkelspårad och enformig. Men självklart håller inte Giuseppe med om det, det finns ju ändå över 3000 olika päron.}
Mr. Andersson, den i framtiden välkända cybertroniska entreprenören. Ingen kan cyborger och biomekaniska implantat som han men det hindrar honom inte från att ständigt arbeta för att förbättra sina kreationer till den grad att han inte bryr sig om vad som är artigt eller moraliskt. Hans nästa idé är att använda rumtidsanomalier för att skapa implantat som är bättre än fysiskt möjligt. Att hitta en anomali är dock mycket svårt eftersom rumtiden är irriterande stabil och det är ju inte direkt så att det ofta sker händelser, typ tidsresor, som stör den. Eller?
}
\blockA{\adjincludegraphics[height=0.27\textheight]{Bilder/Karaktarer/Mr.Andersson.png}} \\ \\

%% Sofonisba Anguissola
\vfil
\blockA{\adjincludegraphics[height=0.27\textheight]{Bilder/Karaktarer/Lisa.png}}
\blockB{\vspace{1ex}\textsc{Lisa}\\
%Sofonisba är en ung men framstående konstnär från Cremona, Italien, som den senaste tiden jobbat för det spanska Hovet. Hennes jobb har blivit lovprisat av Michelangelo själv, men trots det så får hon inte så många möjligheter som konstnär. Hon har alltid svar på tal och hoppas att hon en dag ska få samma möjligheter som hennes manliga kollegor.}
Lisa är en otypisk framtidsmedborgare som arbetar hårt för att försörja sig. För att vara så effektiv som möjligt har Lisa flera olika jobb, men de olika yrkesuppgifterna är inget som gör henne förvirrad. Tvärtom går hon in i sina olika yrkesroller med total hängivelse. Kanske lite för total för att det ska vara psykiskt rimligt, även med framtida mått mätt.}

%% Hildegard
\vfil
%\blockB{\textsc{Hildegard} \\
%Hildegard är en tjänsteflicka vid det kejserliga hovet i Prag. Hon är medveten om alla intriger som sker på hovet, men vet samtidigt att det inte är hennes plats att lägga sig i det. Och även om hon gjorde det så är det ju inte precis någon som bryr sig om vad tjänstefolket tycker eller tänker.}
%Hildegard är en tjänsteflicka vid det kejserliga hovet i Prag. Detta innebär att hon kan allt om vad intriger heter, men inte kan göra något åt dem. Att ingen bryr sig om vad en säger tillåter en viss frispråkighet, vilket Hildegard utnyttjar fullt.}
%\blockA{\adjincludegraphics[height=0.27\textheight]{Bilder/Karaktarer/Hildegard.png}}


\endgroup
