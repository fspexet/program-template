\rubrik{Intresserad av F-spexet?
}
%\phantomsection\addcontentsline{toc}{section}{\sc Intresserad av F-spexet?\dotfill}
%\vspace{-cm}
%\begin{table}
\label{asplapp}

Det finns en massa roliga saker som man kan göra i F-spexet. Man kan stå på scen, bygga
rekvisita, laga mat, sköta hemsidan och mycket mer! \\

Här nedan finns mer utförliga beskrivningar av de olika posterna, så att du vet vad olika grupper har pysslat med under året. Här kan du också ta reda på vilken post som passar just dig, om du vill vara med i nästa års spex!\\

Vill du vara med? QR-koden nedan leder till vår digitala asplapp där ni kan hitta mer information och söka! %<Länk till aspformulär eller sidhänvisning> Mer information om aspningen finns \textbf{\textit{här}}.


\begin{center}
    \includegraphics[width=0.3\textwidth]{Bilder/QR/F-spexet_aspformular.png}
\end{center}
\vspace{0.5cm}
\underrubrik{Grupper i F-spexet
}
\begingroup
\fontsize{11}{12}\selectfont
\parskip=1\baselineskip


{\Large\textit{Koreograf}} \\
När text och musik är fixat, behöver Skådespelarna en dans till dessa också, eftersom det blir lite tråkigt att titta på en person som står stilla och sjunger. Koreografen arbetar med att komma på danser som Skådespelarna ska framföra, och sedan lära Skådespelarna dessa. Detta innebär att ha extra mycket koll på sångerna och melodierna, samt att vara närvarande på Skådespelarnas rep lagom mycket.


{\Large\textit{Kostym}} \\
Det är Kostymgruppen som syr alla fina kläder på scen. Deras ansvar är att leta igenom manuset och bestämma hur Skådespelarna ska se ut, lista ut hur detta kan sys, sy det, och skälla när Skådespelarna har sönder sagda kläder. Ibland syr Kostym även viss tygbaserad rekvisita, exempelvis kuddar och draperier.

{\Large\textit{Kuplettsamordnare}} \\
För att spexet ska vara en musikal och inte bara en teater krävs det låtar och låttexter, i spexsammanhang kända som kupletter och kuplettexter. Kuplettsamordnaren ser till att kupletterna har vettiga och roliga melodier, att intresserade spexare skriver kuplettexterna, och att kupletterna blir färdiga i tillräckligt god tid för att Skådespelarna ska hinna lära sig dem.


{\Large\textit{Ljud och Ljus}} \\
LoL har till uppgift att bestämma vilken typ av ljud- och ljusutrustning årets spex behöver samt rigga denna innan föreställningarna. De fixar också ljudeffekter till spexet samt sköter reglagen under föreställningarna för att de som ska synas syns och de som ska höras hörs.


{\Large\textit{Manusförfattare}} \\
En av de mest essentiella ingredienserna för ett bra spex är ett bra manus. Manusgruppen är den grupp som innan andra spexare ens har kommit att tänka på nästa års spex redan har bestämt tema, synopsis, karaktärer, repliker, nödvändig rekvisita, lagt förslag på kupletter och en massa annat.

{\Large\textit{Mat}} \\
Matgruppen har som huvudsaklig uppgift att se till att det serveras en trerätters middag under föreställningarna. Detta innebär i realiteten att planera inköp, handla samt anpassa maten till olika allergier och matpreferenser. Förutom föreställningarna står Matgruppen även för maten på sittningar och dylikt inom spexet. Dessa processer förenklas något av att det finns spexare att tillgå under matlagningen som gör det faktiska hackandet, stekandet och kokandet enligt Matgruppens instruktioner.

{\Large\textit{Media}} \\
Mediagruppen ser till att folk vet vad spexet håller på med, ansvarar för all reklam såsom biljetter och affischer, gör programmet till föreställningarna, sköter spexets hemsida och sociala medier, samt dokumenterar såväl föreställningarna som arbetet under året. Media är lite av en paraply-grupp, så här passar många intressetyper in. Design, foto, film, kodande, PR, rebusskapande, skrivande och att beställa roliga saker från nätet är alla uppgifter som ryms inom Mediagruppen.

\newpage

{\Large\textit{MåBra}} \\
MåBra har till uppgift att övriga spexare (och de själva) mår bra, har det så roligt som möjligt och gör sitt bästa. För att uppnå sitt mål brukar de använda sig av ett hemligt vapen gemenligen kallat ``kakor”. MåBra-gruppen har ett antal fasta arbetsuppgifter: de sköter baren under föreställningarna, fixar pizza och liknande åt spexarna när spexet har jobbat hela dagen och fixar fika till vissa spexaktiviteter. Utöver detta kan de baka kakor, fixa peppiga spellistor, ha en massage-station... Det finns ingen gräns för hur bra spexarna kan må!


{\Large\textit{Orkester}} \\
Det är Orkestern som förgyller föreställningarna med fantastisk live-musik. Förutom att arrangera kupletter och spela upp dessa, spelar de också ett aktintro till varje akt. En bra färdighet i Orkestern är, förutom att så klart kunna spela sitt instrument, att arrangera musik. Orkestern repar vanligtvis mellan två och tre gånger i veckan.


{\Large\textit{Producent}} \\
Producenten är spexets allmänt konstnärliga ledare för hela föreställningen. Det innebär att producenten ser till att spexet har en enhetlig konstnärlig vision, och har på så sätt ett finger med i såväl rekvisita som kupletter, och samordnar mellan grupperna överlag.


{\Large\textit{Regissör}} \\
Regissörens uppgift är att leda Skådespelarnas rep, samt gnälla och tjata på Skådespelarna så att de gör rätt saker vid rätt tillfälle under dessa rep (naturligtvis med en grund i årets manus). Som Regissör är man också ansvarig för att det blir rep lagom ofta, både för Skådespelarna själva och tillsammans med Orkestern.


{\Large\textit{Scen}} \\
Varje spexmanus kräver en mängd dekor och rekvisita. Det kan var allt från böcker eller papper till nycklar och kanoner, lönndörrar, och diverse sorters maskiner. Scengruppen
behöver använda sin uppfinningsrikedom för att snickra, pyssla, bygga, och måla allt detta. Scengruppens ansvar är också att se till så att det existerar en faktisk scen att spela upp spexet på och kulisser att spela upp spexet framför. Under föreställningarna står Scengruppen redo att diskret ge rätt Skådespelare rätt sak vid rätt tillfälle, eller förflytta
rekvisita enligt manuset.

{\Large\textit{Skådespelare}} \\
Även kallad Ensemblen. Skådespelarnas uppgift i spexet är att stå på scen och framföra det spex som manusgruppen har drömt upp, regissören regisserat och publiken bett om. Den som blir invald som Skådespelare får en roll tilldelad sig, och sätter därefter igång med att lära sig manuset, sångtexter och dansstegen utantill. De övar också på improvisation inför inropen. Skådespelarna repar under ledning av Regissören, både på egen hand och tillsammans med Orkestern. Det brukar bli i genomsnitt tre kvällar i veckan under året.


{\Large\textit{Smink}} \\
Detta är gruppen som ger spexare färg och gör så att Skådespelarna ser ut som sina karaktärer med hjälp av peruker, smink, träben, huggtänder och dylikt. De flesta Skådespelare lär sig applicera rätt smink själva, men Sminkgruppen ska vara beredda på att få måla mycket i andras ansikten.

{\Large\textit{Styret}} \\
Styret är de som ansvarar för att det överhuvudtaget blir ett F-spex. De sitter i otaliga möten för att planera, boka, styra, och se till att spexet fungerar och är roligt. 
Ordföranden, som även går under namnet Tant, leder styrets arbete och är F-spexets ansikte utåt. 
Förmannen, vanligen kallad Fröken, är spexets arbetsledare som ser till att varje grupp arbetar och gör det de ska, samt planerar och leder spexets aktiviteter. 
Kassören, även kallad Fru, har ansvar för spexets ekonomi, och uppgifterna utgörs framförallt av att sköta bokföringen och ansvara för biljettförsäljningen. 
De övriga posterna är ledamöter. För alla stora och små uppgifter som inte täcks av de andra posterna finns det alltid en ledamot som kan ställa upp och föra spexets arbete framåt.


\endgroup