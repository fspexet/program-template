\begingroup
\newcommand\blockA[1]{\parbox[c][0.23\textheight][c]{0.3\textwidth}{\centering #1}}
\newcommand\blockB[1]{\parbox[c][0.23\textheight][c]{0.65\textwidth}{\raggedright #1}}

% Johannes Kepler
\blockA{\adjincludegraphics[height=0.27\textheight]{Bilder/Karaktarer/Einstein.png}}
\blockB{\textsc{Albert Einstein}\\
%Johannes är en ung astronom som spenderat sina dagar vid universitetet i Tübingen under skolning av Michael Mästlin. Han beundrar sin mentor Michael och vill göra honom stolt, men samtidigt stå på egna fötter. Han vill lära sig allt om allt och strävar efter att utveckla den vetenskap han älskar mest; astronomi.}
Albert Einstein i egen kort person. På senare år har den excentriska vetenskapsmannen dragit sig tillbaka från omvärldens rampljus för att ägna sig åt än mer excentriska idéer och uppfinningar. Om det är något som han har lärt sig under sitt långa liv är det att han alltid vet bäst, oavsett om han faktiskt har rätt eller inte. Alla hans resonemang är fullt underbyggda av logik och fysik medan andras åsikter och rimlighet spelar en mindre roll i tankeprocessen.}

% Michael Mästlin
\vfil
\blockB{\vspace{2ex}\textsc{Karl ``Kalle´´ Andersson}\\
%Michael är en bitter gammal gubbe med sin storhetstid bakom sig. Han har jobbat i många år vid universitet i Tübingen, men har nu fått en chans att hjälpa sin protogé, Johannes Kepler, uppåt i världen; och detta tänker han inte gå miste om. Han känner sedan tidigare Tycho och de har haft sina meningsskillnader, då han anser att Tychos mätningar inte står en chans mot hans teoretiska modeller.}
Kalle är Einsteins kollega samt en ung och lovande forskare med skarp blick och engagemang. Har han något som han vill säga drar han sig inte en sekund för att göra sin röst hörd -- om Einstein lyssnar på honom är dock en annan fråga. Kalle är en självsäker individ som vet vad han vill, såväl i forskning som i kärlek.}
\blockA{\adjincludegraphics[height=0.27\textheight]{Bilder/Karaktarer/Kalle.png}} \\ \\

%% Tycho Brahe
\vfil
\blockA{\adjincludegraphics[height=0.27\textheight]{Bilder/Karaktarer/Laplace.png}}
\blockB{\vspace{1ex}\textsc{Laplace}\\
Laplace är tvillingbror till Nabla och lite av en ensamvarg. Han brukar hänga vid sin verkstad och laga saker i lugn och ro, helst utan att prata för mycket med främlingar, men när tidsresenärer dyker upp är det svårt att inte vara nyfiken. Särskilt  när en av dem är så snygg och charmig (och nej, det är inte Einstein som avses). Nämnde vi förresten att han är en cyborg?}
%Tycho är en glad men utarbetad man på ålderns höst. Som långvarig hovastronom vet han var saker och planeter hör hemma och har inte tid för nymodigheter som heliocentrismen. Han störs ofta i sitt arbete av kejsaren som bryr sig mer om astrologi än astronomi!}

%% Rudolf II von Habsburg
\vfil
%\blockB{\textsc{Rudolf II von Habsburg} \\
%Rudolf är kejsare över det Tysk-Romerska riket och sitter på sin tron i det kejserliga palatset i Prag. Rudolf omger sig av vetenskapsmän och konstnärer, där hans favorit är hans bästa vän Giuseppe, och hade helst lämnat de tråkiga sakerna, som att styra riket, till andra och låter hellre ödet avgöra vad han ska göra. Han gillar konst och vetenskap, men han förstår sig inte på det själv.}
%Rudolf är kejsare över det Tysk-Romerska riket och sitter på sin tron i det kejserliga palatset i Prag. Rudolf omger sig av vetenskapsmän och konstnärer, där hans favorit är hans bästa vän Giuseppe, och hade helst lämnat de tråkiga sakerna, som att styra riket, till andra och låter hellre ödet avgöra vad han ska göra. Han gillar konst och vetenskap, men han förstår sig inte på det själv.}
%\blockA{\adjincludegraphics[height=0.27\textheight]{Bilder/Karaktarer/Rudolf.png}}


\endgroup
